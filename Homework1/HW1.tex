\documentclass[11pt]{article}

\usepackage{amsmath}
\usepackage{geometry}
\geometry{
 a4paper,
 textwidth = 500pt
}
\usepackage{hyperref}


\begin{document}

\title{Homework 1 - Data Mining - Sapienza}
\author{Ivan Fardin 1747864}
\date{November 1$^{st}$, 2020}

\maketitle

\tableofcontents

\newpage
\section{Problem 1}

\subsection{}
sample space $\Omega$ = \{ \newline
                \hspace*{1cm}   2 of Hearts, 3 of Hearts, 4 of Hearts, 5 of Hearts, 6 of Hearts, 7 of Hearts, 8 of Hearts, 9 of Hearts, \newline
                \hspace*{1cm}   10 of Hearts, J of Hearts, Q of Hearts, K of Hearts, A of Hearts, \newline
                \hspace*{1cm}   2 of Diamonds, ..., A of Diamonds, \newline
                \hspace*{1cm}   2 of Spades, ..., A of Spades, \newline
                \hspace*{1cm}   2 of Clubs, ..., A of Clubs \}

\bigskip
\noindent
$ \displaystyle |\Omega| = 13 \cdot 4 = 52$    \hspace{1cm} (13 cards for each suit)

\bigskip
\noindent
Pr(2 of Hearts) = ... = Pr(A of Clubs) = $ \displaystyle \frac{1}{|\Omega|} = \frac{1}{52}$

\bigskip
\noindent
Pr("Pick a 2 in the deck") = ... = Pr("Pick an ace in the deck") = $ \displaystyle \frac{4}{52} = \frac{1}{13}$

\bigskip
\noindent
Pr("Pick a Hearts in the deck") = ... = Pr("Pick a Clubs in the deck") = $ \displaystyle \frac{13}{52} = \frac{1}{4}$

\subsection{}

\subsubsection{}
Pr("Pick at least an ace in the first three cards") = $ \displaystyle  1 - \frac{\binom{4}{0} \ \binom{48}{3}}{\binom{52}{3}} =
1 - \frac{48! \ 49!}{45! \ 52!} \approx 1 - 0.78 = 0.22 $

\bigskip
We have $ \displaystyle \binom{48}{3}$ combinations to pick 3 cards all of different rank from the ace out of $ \displaystyle \binom{52}{3}$ possible hands.
Since we are looking for at least an ace, we can compute the complement of this probability.

\subsubsection{}
Pr("Pick exactly an ace in the first five cards") = $ \displaystyle \frac{\binom{4}{1} \ \binom{48}{4}}{\binom{52}{5}} =
\frac{48! \ 5! \ 47!}{3! \ 44! \ 52!} \approx 0.3 $

\bigskip
We have $ \displaystyle \binom{4}{1}$ combinations to pick an ace and $ \displaystyle \binom{48}{4}$ to pick four other cards of a different rank out of
$ \displaystyle \binom{52}{5}$ possible hands.


\subsubsection{}
Pr("Pick the first three cards of the same rank") = $ \displaystyle \frac{13 \ \binom{4}{3}}{\binom{52}{3}} = \frac{13 \cdot 4! \ 49!}{52!} \approx 0.0024 $

\bigskip
We have 13 ways to make three of a kind ($ \displaystyle \binom{4}{3}$ combinations) out of $ \displaystyle \binom{52}{3}$ possible hands.


\subsubsection{}
Pr("Pick all Diamonds in the first five cards") = $ \displaystyle \frac{\binom{13}{5}}{\binom{52}{5}} = \frac{13! \ 47!}{52! \ 8!} \approx 0.0005 $

\bigskip
We have $ \displaystyle \binom{13}{5}$ combinations to pick 5 out of 13 Diamonds out of $ \displaystyle \binom{52}{5}$ possible hands.


\subsubsection{}
Pr("Pick a full house in the first five cards") = $ \displaystyle \frac{13 \ \binom{4}{3} \ 12 \ \binom{4}{2}}{\binom{52}{5}} =
\frac{13 \cdot 12 \cdot 4! \ 5! \ 47!}{52!} \approx 0.0014 $

\bigskip
We have 13 ways to make three of a kind ($ \displaystyle \binom{4}{3}$ combinations) and 12 ways to constitute a pair among the remaining ranks
($ \displaystyle \binom{4}{2}$ combinations) out of $ \displaystyle \binom{52}{5}$ possible hands.


\subsection{}
I developed a simple Python program called $cards.py$ to perform simulations of picking the first five cards of a deck in order to check my answers.

\bigskip
It makes use of two arrays: one for the card numbers and the other for the four suits. \newline
So, it simply picks one card at a time from the deck by randomly choosing two items from the two arrays
and if the card is not already drawn prints it, otherwise picks a new one until a non-drawn card is printed.

\newpage
\section{Problem 2}
sample space $\Omega$ of the sum of $n=3$ regular dice = \{ n, ..., 6n \} = \{ 3, ..., 18 \}

\bigskip
\noindent
sample space $\Omega'$ of throwing $n=3$ regular dice = \{ $(a,b,c) : 1 \leq a,b,c \leq 6$ \} = \{ (1,1,1), ..., (6,6,6) \}

\bigskip
\noindent
$ \displaystyle |\Omega| = 6n - n = 5n = 15 $

\bigskip
\noindent
$ \displaystyle |\Omega'| = 6^n = 6^3 $

\bigskip
To calculate the probability of seeing a sum of 11 or a sum of 16, we need to partition the two numbers into the sum of three integers
in the range from 1 to 6 (faces of a die).

\bigskip
$ 11 = 6+4+1 = 6+3+2 = 5+5+1 = 5+4+2 = 3+5+3 = 3+4+4 $ \newline
\indent
$ 16 = 6+6+4 = 6+5+5 $

\bigskip
The number of permutations for each partition is $ \displaystyle \frac{n!}{n_1! \ n_2! \ n_3!} $ where:
\begin{itemize}
  \item $n$ is the number of dice,
  \item $n_1$ is the number of repetition of the same value in the sum different from $n_2$ and $n_3$,
  \item $n_2$ the number of repetition of the same value in the sum different from $n_1$ and $n_3$,
  \item $n_3$ is the number of repetition of the third value in the sum different from $n_1$ and $n_2$
\end{itemize}

\bigskip
The number of permutations of the partitions of 11 is $ \displaystyle 3! + 3! + \frac{3!}{2!} + 3! + \frac{3!}{2!} + \frac{3!}{2!} = 3 \ (3! + 3) = 27 $

\bigskip
The number of permutations of the partitions of 16 is $ \displaystyle \frac{3!}{2!} +\frac{3!}{2!} = 2 \times 3 = 6 $

\bigskip
Pr("Sum of 11") = $ \displaystyle \frac{27}{6^3} = 0.125 $

\bigskip
Pr("Sum of 16") = $ \displaystyle \frac{6}{6^3} = \frac{1}{6^2} = 0.02\overline{7} $

\bigskip
Pr("Sum of 11" $\cup$ "Sum of 16") = Pr("Sum of 11") + Pr("Sum of 16") = $ 0.125 + 0.02\overline{7} = 0.152\overline{7} $

\newpage
\section{Problem 3}
sample space $\Omega$ = \{ TP, FP, TN, FN \} = \{ $(P_t,P), (P_t,N), (N_t,N), (N_t,P)$ : \newline
  \hspace*{1cm} each pair represents (result outcome, person infected) \}

\bigskip
\noindent
$ \displaystyle |\Omega| = 4 $

\bigskip
\noindent
Sensitivity of the rapid test = $ \displaystyle \frac{TP}{TP+FN} = 84.7\% $ \newline \newline
Specificity of the rapid test = $ \displaystyle \frac{TN}{TN+FP} = 85.7\% $

\bigskip
\noindent
Let's assume 1\% of the population currently has COVID-19 $\Rightarrow$ Pr("person is infected") = Pr(P) = 1\%

\bigskip
\noindent
Result of the rapid test = positive $\Rightarrow$ probability that this person is infected with COVID-19?

\bigskip
\bigskip
\noindent
The result can be either a TP or an FP $\Rightarrow$ Pr($P_t$) = Pr(TP) + Pr(FP)

\bigskip
\noindent
Pr("positive test given person is infected") = Pr($P_t | P$) = Sensitivity \newline
Pr("negative test given person is not infected") = Pr($N_t | N$) = Specificity

\bigskip
\noindent
$\Rightarrow$

\bigskip
\noindent
Pr(TP) = Pr($(P_t,P)$) = Pr($P_t | P$) Pr(P) $\iff$ Pr($P_t | P$) = $ \displaystyle\frac{Pr(P_t \cap P)}{Pr(P)} = \frac{Pr(P|P_t) Pr(P_t)}{Pr(P)}$ = 84.7\% \newline

\noindent
Pr(TN) = Pr($(N_t,N)$) = Pr($N_t | N$) Pr(N) $\iff$ Pr($N_t | N$) = 85.7\% \newline

\noindent
Pr(FP) = Pr($(P_t,N)$) = Pr($P_t | N$) Pr(N) $\iff$ Pr($P_t | N$) = 1 - Specificity = 14.3\% \newline
Pr(FN) = Pr($(N_t,P)$) = Pr($N_t | P$) Pr(P) $\iff$ Pr($N_t | P$) = 1 - Sensitivity = 15.3\%

\bigskip
\noindent
$\Rightarrow$ Pr($P|P_t$) = $ \displaystyle\frac{Pr(P_t|P) Pr(P)}{Pr(P_t)} = \frac{Pr(P_t|P) Pr(P)}{Pr(P_t|P) Pr(P) + Pr(P_t|N) Pr(N)} =
\frac{0.847 \cdot 0.01}{0.847 \cdot 0.01 + 0.143 \cdot 0.99} = \\
= 0.0565 = 5.65\% $

\bigskip
The probability that a person is infected with COVID-19 given test is positive can be calculated using Bayes theorem. \newline
The fact that the probability of positive tests is the sum between the probability of being either a TP or an FP
is an application of the Law of Total Probability.

\bigskip
The probability that a person is infected with COVID-19 given test is positive is strongly affected by the assumption that only 1\% of the population
currently has COVID-19. If we assume that not 1\% but 10\% of the population currently has COVID-19, the probability rises to $\approx 40\%$,
and if we assume 15\% it becomes 51\%. \newline
So, I think that such a test can be useful along with other more accurate tests.


\newpage
\section{Problem 4}
\subsection{}
sample space $\Omega$ of the $G_{n,p}$ model = \{ $(v_i, v_j)$ : 1 $\leq i < j \leq n$ and $(v_i, v_j) \in M$ with probability $p$ \} \newline
where $n$ is the number of nodes, $M$ is the set of edges and $(v_i, v_j) = (v_j, v_i)$ so I count it once.

\bigskip
\noindent
The maximum number of edges is equal to the number of edges in a complete graph $ \displaystyle \binom{n}{2} = \frac{n \ (n-1)}{2} $

\bigskip
\noindent
$ \displaystyle |\Omega| = m = \binom{\frac{n \ (n-1)}{2}}{m} \ p^m \ (1-p)^{\frac{n \ (n-1)}{2}-m} $   \hspace{1cm} (number of edges)


\subsection{}
Pr($(v_i, v_j) \in M$) = $p$


\subsection{}
Suppose $n$ even, find the probability that the graph contains exactly two cycles of size $\frac{n}{2}$ and no other edges.
The number of edges must be $n$ since each node must be connected to two: the previous and next.

\bigskip
\noindent
Pr("graph has exactly $n$ edges") = $ \displaystyle \binom{\frac{n \ (n-1)}{2}}{n} \ p^n \ (1-p)^{\frac{n \ (n-1)}{2} \ -n} =
\frac{1}{n \ (n-3)!} \ p^n \ (1-p)^{\frac{n \ (n-3)}{2}} $

\bigskip
Furthermore, there are no nodes or edges in common between the two cycles otherwise the two cycles would not be the same size or a third cycle would exist.
A cycle is a sequence of vertices in which each node appears once \newline
$v_1, ..., v_{l_1}, v_1$ where $l_1 = \frac{n}{2}$, with $v_i \neq v_j$ for $i,j \in \{ 1, ..., l_1 \} = M_1$ \newline
$v'_1, ..., v'_{l_1}, v'_1$ where $l_1 = \frac{n}{2}$, with $v'_i \neq v'_j$ for $i,j \in \{ 1, ..., l_2 \} = M_2$ \newline
$l_1 + l_1 = n$ and $(v_i, v_{i+1}) \neq (v'_i, v'_{i+1})$ for all edges $\in M_1, M_2$

\bigskip
There exist $ \displaystyle \binom{n}{\frac{n}{2}} $ combinations of $\frac{n}{2}$ items out of $n$.
Each combination denotes a partition of the graph into two sets, we need exactly $n$ edges to connect the $n$ nodes and create two components

\bigskip
\noindent
Pr("graph contains exactly two cycles of size $\frac{n}{2}$ and no more edges") =
$ \displaystyle \binom{n}{\frac{n}{2}} \ \binom{\frac{n \ (n-1)}{2}}{n} \ p^n \ (1-p)^{\frac{n \ (n-1)}{2} - n} $


\subsection{}
Unlike \textbf{4.3} where we only take partitions of size $\frac{n}{2}$, here we need to take also partitions that make the sum equal to $n$ with a minimum
number of items of a partition equal to three (a cycle has at least 3 nodes).

\bigskip
\noindent
Pr("graph contains exactly two cycles of any size with no common nodes and no more edges") = \newline
$ = \displaystyle \sum_{i=3}^{\frac{n}{2}} \binom{n}{i} \binom{n}{n-i}\ \binom{\frac{n \ (n-1)}{2}}{n} \ p^n \ (1-p)^{\frac{n \ (n-1)}{2} - n} $


\subsection{}
The probability a vertex has degree $k$ is a Bernoulli Random Variable
$$P(X = deg(v) = k) = \binom{n-1}{k} \ p^k \ (1-p)^{n-1-k} $$

\bigskip
We have $ \displaystyle \binom{n-1}{k}$ ways of choosing $k$ edges out of the $n-1$ possible edges,
$p^k$ is the probability that the $k$ edges are present and
$p^{n-k-1}$ is the probability that the remaining $n-k-1$ edges are not.

\bigskip
So, the expected degree of the vertex $i$ is the expected value of a Bernoulli Random Variable \newline

\begin{equation}
    \label{X_ij}
    X_{ij} =
    \begin{cases}
      1 \qquad if \ the \ edge \ (v_i,v_j) \in M \\
      0 \qquad otherwise
    \end{cases}
\end{equation}

$$ E[X_{ij}] = 0 \cdot Pr(X_{ij} = 0) + 1 \cdot Pr(X_{ij} = 1) = p $$
$$ E[X] = \sum_{k=0}^{n-1} k \cdot Pr(X=k) = E[\sum_{j=1}^{n-1} X_{ij}] = \sum_{j=1}^{n-1} E[X_{ij}] = p \ (n-1) $$


\subsection{}
From \textbf{4.1}, \textbf{4.2} and \textbf{4.5} the expected number of edges is the expected value of Bernoulli Random Variable

\bigskip
\noindent
$$E[Y] = \sum_{m=0}^{\frac{n \ (n-1)}{2}} m \cdot Pr(Y=m) = E[\sum_{m=0}^{\frac{n \ (n-1)}{2}} X_{ij}] = \sum_{m=0}^{\frac{n \ (n-1)}{2}} E[X_{ij}] =
p \ \frac{n \ (n-1)}{2} $$


\subsection{}


\newpage
\section{Problem 5}
To find the 10 beers with the highest number of reviews from the \textit{beers.txt} file,
I wrote a simple bash script called \textit{beers.sh} where I chained the following Unix commands through pipes, so that the output text of each process (stdout) is passed directly as input (stdin) to the next one.

\begin{verbatim}
cut -f 1 beers.txt | sort | uniq -c | sort -k 1 -t$'\t' -nr | head
\end{verbatim}

\bigskip
I assumed that each line in the \textit{beers.txt} file contains a review consisting of the beer name and the associated score. \newline

\bigskip
From the \textit{man} command:
\begin{itemize}
  \item \begin{verbatim} cut -f 1 beers.txt \end{verbatim} Selects only the first field (i.e. beer name, fields separated by tab) of the $beers.txt$ file
  \item \begin{verbatim} sort \end{verbatim} Sorts the beer names alphabetically
  \item \begin{verbatim} uniq -c \end{verbatim} Filters adjacent matching lines from standard input, writing to standard output (sort the input first)
    for each beer name a prefix denoting the number of occurrences in the input
  \item \begin{verbatim} sort -k 1 -t$'\t' -nr b.txt \end{verbatim} Sorts the standard input in reverse order according to string numerical value
    of the first column (separated by the tab character from the second one)
  \item \begin{verbatim} head \end{verbatim} Prints the first 10 lines of the standard input to standard output
\end{itemize}


\newpage
\section{Problem 6}
To find the top-10 beers with the highest average overall score among the beers that have had at least 100 reviews,
I wrote a simple Python program called \textit{beers.py}.

\bigskip
Since some beers appear in multiple lines of the \textit{beers.txt} file, preprocessing is needed in order to sum the score of reviews
referring to the same beer.
So, I thought about using a hash table (in Python, the dictionary is implemented as a hash table)
where each key is represented by a beer name and the value by a pair consisting of the overall score and the number of reviews associated with the beer.

\bigskip
After that, as in \textbf{Problem 5}, I read each line of the \textit{beers.txt} file, split it according to the tab character
then I added the score of the review to the overall score so far and I updated the number of reviews.

\bigskip
In the end, I sorted the dictionary by the average overall score of each entry within the dictionary from largest to smallest via the sorted function
and printed the top 10 beers that have had at least 100 reviews.


\newpage
\section{Problem 7}
\subsection{}
As suggested, I used \textit{Twython} to get the stream of the tweets in Rome as they are generated.

\bigskip
First, I wrote the \textit{twitter-config.json} file where I stored sensitive information. \newline
Then, in \textit{Auth.py} I wrote a class to retrieve information from the above file.

\bigskip
After this initial step, in \textit{RomeStreamTweets.py} I start two threads: a listener and a writer and wait for \textit{Ctrl+C} command to properly close
the program.

\bigskip
The listener code is implemented in \textit{TwitterListener.py} where a Streamer object is created and a filtered query using a bounding
box for Rome location is performed.

\bigskip
The streamer (implemented in \textit{Streamer.py}) is a \textit{TwythonStreamer} that receives tweets from the \textit{Twitter Streaming API} and extract from them information of interest
for the application like the username, the text and the exact geographic location (if any otherwise the place).
When all the information is present, a minimalist version of the tweet is added to a list.

\bigskip
The writer code is implemented in \textit{TwitterWriter.py} where the same list is continuously written into a JSON file called \textit{tweets.json}.


\subsection{}
To plot the tweet locations on Google Maps, I wrote a simple HTML page \textit{RomeStreamTweetsMap.html}
where based on the points saved in the \textit{tweets.json} file, it puts markers on the map of Rome where the tweets are located.

\bigskip
To avoid explicitly inserting the Google Maps API key into the code via the following script \newline

\textit{\textless script src="https://maps.googleapis.com/maps/api/js?key=API\_KEY\& \newline
callback=initMap\&libraries=\&v=weekly" defer\textgreater \textless /script\textgreater} \newline

I wrote a \textit{gmaps-config.json} file where I stored the key and read it from there by dynamically loading the map.

\bigskip
As workaround for the cross-origin policy issue, I used Python to start a web server on the local machine via
\textit{python3 -m http.server} for loading the json file containing the tweets and then I entered the following URL in a browser
\url{http://localhost:8000/RomeStreamTweetsMap.html}.

\bigskip
So, to start capturing tweets and plotting them on the map, you need to launch the Python program via \textit{python3 -m RomeStreamTweets.py} and
access the above URL.


\end{document}
